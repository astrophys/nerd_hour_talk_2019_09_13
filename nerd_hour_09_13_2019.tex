% Author : Ali Snedden
% Date   : 9/10/19
% Compile:
%   pdflatex -shell-escape nerd_hour_09_13_2019.tex


\documentclass{beamer}
%\usetheme{Ilmenau}
\usetheme{Copenhagen}
%\usecolortheme{seahorse}
% See http://www.hartwork.org/beamer-theme-matrix/ for more themes and colors

% for themes, etc.
%\mode<presentation>{ \usetheme{boxes} }
%\mode<presentation>{ \usetheme{Warsaw} }
%\renewcommand*{\bibfont}{\scriptsize}

\usepackage{times}  % fonts are up to you
\usepackage{graphics}
\usepackage{amsmath}
\usepackage[makeroom]{cancel}
\usepackage{media9}
\usepackage{hyperref}
\usepackage{movie15}
\usepackage{epstopdf}

\usepackage{tikz}
\usetikzlibrary{fit}

\usepackage[absolute,overlay]{textpos}
\usepackage{psfrag}

%http://tex.stackexchange.com/questions/302584/problems-related-to-beamer-footcite
\makeatletter
\def\@iiiminipage#1#2[#3]#4{%
    \leavevmode
    \@pboxswfalse
    \setlength\@tempdima{#4}%
    \def\@mpargs{{#1}{#2}[#3]{#4}}%
    \setbox\@tempboxa\vbox\bgroup
    \color@begingroup
    \hsize\@tempdima
    \textwidth\hsize \columnwidth\hsize
    \@parboxrestore
    \let\@footnotetext\@mpfootnotetext
    \let\@listdepth\@mplistdepth \@mplistdepth\z@
    \@minipagerestore
    \@setminipage}
\makeatother

%http://tex.stackexchange.com/questions/138744/how-do-i-remove-citation-number-from-footcite
%\newcommand\footcitenonum[1]{%
%  \begingroup
%  \renewcommand\thefootnote{}\footnote{#1}%
%  \addtocounter{footnote}{-1}%
%  \endgroup
%}
%
%\makeatletter
%\renewrobustcmd{\blx@mkbibfootnote}[2]{%
%  \iftoggle{blx@footnote}
%    {\blx@warning{Nested notes}%
%     \addspace\mkbibparens{#2}}
%    {\unspace
%     \ifnum\blx@notetype=\tw@
%       \expandafter\@firstoftwo
%     \else
%       \expandafter\@secondoftwo
%     \fi
%       {\csuse{blx@theendnote#1}{\protecting{\blxmkbibnote{end}{#2}}}}
%       {\csuse{footnotenonum#1}{\protecting{\blxmkbibnote{foot}{#2}}}}}}
%\makeatother

%Bibliography stuff
%\usepackage[style=verbose,autocite=footnote,maxnames=10,babel=hyphen,hyperref=true,abbreviate=false,backend=biber,mcite]{biblatex}

\usepackage[style=authoryear]{biblatex}
%\usepackage[bottom]{footmisc}
%\bibliography{/Users/asnedden/Library/texmf/bibtex/bib/references}
\addbibresource{/Users/asnedden/Library/texmf/bibtex/bib/references.bib}
%\bibliographystyle{apj}
\setbeamertemplate{bibliography item}[text]
%\usepackage[backend=bibtex, style=authoryear]{biblatex}
%\addbibresource{/Users/ali/Library/texmf/bibtex/bib/references.bib}
\newcommand{\customcite}[1]{\citeauthor{#1}, \citeyear{#1}}
\renewcommand*{\bibfont}{\tiny}

\usefonttheme[onlymath]{serif}

\newcommand\smallFont{\fontsize{8}{7.2}\selectfont}   %Change font size.
\newcommand\mCite[1]{[\cite{#1}, \citetitle{#1}]}  %Prints name and title
\newcommand\FrameText[1]{
\begin{textblock*}{\paperwidth}(0pt,\textheight)
    \vspace{1.0cm}
    \raggedleft \smallFont #1
\end{textblock*}}


%Get rid of ugly copenhagen default symbol for enumerate
\setbeamertemplate{enumerate items}[default]   

%Increase default separation size between items in itemize and enumerate 

\tikzset{type1/.style={rectangle, rounded corners, minimum width=3cm, minimum height=0.1cm,text centered, draw=black, fill=blue!10, text width=2cm},
        type2/.style={rectangle, rounded corners, minimum width=3cm, minimum height=0.1cm, text centered, draw=black, fill=blue!10, text width=5cm},
        info/.style = {rectangle, rounded corners, minimum width=2.5cm, minimum height=0.1cm, text centered, draw=black, 
            fill=blue!30, text width=2.5cm},
        org/.style={rectangle, rounded corners, minimum width=2cm, minimum height=0.1cm, text centered, draw=black, 
            fill=blue!30, text width=3cm},
        decision/.style = {square, minimum width=1cm, minimum height=0.1cm, text centered, 
            draw=black, fill=green!30, text width=4cm},
        arrow/.style = {thick,->,>=stealth},
    }
    
    
    
   

%%%%%%%%%%%%%%% BIBLIOGRAPHY COMMANDS %%%%%%%%%%%%%%%
\newcommand{\apj}{ApJ.}
\newcommand{\apjs}{ApJS}
\newcommand{\apjl}{ApJL}
\newcommand{\pasp}{Publications of the Astronomical Society of the Pacific}
\newcommand{\pasj}{Publ. of the Astron. Soc. of Japan}
\newcommand{\prc}{Phys. Rev. C}
\newcommand{\prd}{Phys. Rev. D}
\newcommand{\nat}{Nature}
\newcommand{\mnras}{Mon. Not. R. Astron. Soc.}
\newcommand{\na}{New Astronomy}
\newcommand{\aap}{Astronomy and Astrophysics}
\newcommand{\araa}{Annu. Rev. Astron. Astrophys}
\newcommand{\aj}{Astronomical Journal}
\newcommand{\pasa}{Pub. of the Astron. Soc. of Australia}
\newcommand{\nar}{New Astronomy Reviews}


%%%%%%%%%%%%%%% OTHER COMMANDS %%%%%%%%%%%%%%%%%
\newcommand{\subitem}{\item[$-$]}


% these will be used later in the title page
\title{Ripples in the Universe : Black hole and Neutron Star mergers}
\author{Ali Snedden
}
\date{September 13, 2019}
%\subtitle{stuff}

% note: do NOT include a \maketitle line; also note that this title
% material goes BEFORE the \begin{document}

% Recurring Outline
\AtBeginSection[]  % "Beamer, do the following at the start of every section"
{
\begin{frame}<beamer> 
\frametitle{Outline} % make a frame titled "Outline"
\tableofcontents[currentsection]  % show TOC and highlight current section
\end{frame}
}

\begin{document}

% this prints title, author etc. info from above
\begin{frame}
\titlepage
\end{frame}





%%%%%%%%%%%%%%%%%%%%%%%%%%%%%%%%%%%%%%%%%%%%%%%%%%%%%%%%%
%%%%%%%%%%%%%   Introduction to Stars  %%%%%%%%%%%%%%%%%%
%%%%%%%%%%%%%%%%%%%%%%%%%%%%%%%%%%%%%%%%%%%%%%%%%%%%%%%%%
%\section{The time before the Universe...}

%
%
\section{What is a Star?}

\begin{frame}
\frametitle{What is a Star?}
\begin{picture}(320,250) 
    % http://large.stanford.edu/courses/2011/ph241/olson1/
    \put(-35, 50){\includegraphics[height=2.15in]{images/force_diagram.eps}}
    \put(130, 220){\begin{minipage}[t]{0.7 \linewidth}
    \begin{itemize}
        \item Gas - primarily hydrogen and helium
        \bigskip
        \pause
        \item Gravitationaly bound 
        \bigskip
        \pause 
        \item Dense enough for nuclear fusion 
    \end{itemize}
    \end{minipage}}
\end{picture}
\end{frame}


\begin{frame}
\frametitle{What is Nuclear Fusion?}
\begin{picture}(320,250) 
    \put(-20, 78){\includegraphics[height=2.25in]{images/Hcycle.eps}}
%    \put(120, 240){\begin{minipage}[t]{0.7 \linewidth}
%    \begin{itemize}
%    \end{itemize}
%    \end{minipage}}
\end{picture}
\end{frame}

\begin{frame}
\frametitle{Life Cycle of a Star}
Fate determined by :
\bigskip
\begin{itemize}
    \item Mass
    \pause
    \bigskip
    \item Chemical composition
    \pause
    \bigskip
    \item Binary Status
\end{itemize}
\end{frame}

\begin{frame}
\frametitle{Life Cycle of a Star}
\begin{picture}(320,250) 
    %http://astronomy.swin.edu.au/cosmos/H/Hertzsprung-Russell+Diagram
    %By ESO - https://www.eso.org/public/images/eso0728c/, CC BY 4.0, https://commons.wikimedia.org/w/index.php?curid=19915788
    %\put(30, 38){\includegraphics[height=3in]{/Users/asnedden/Projects/NCH/Documents/Talks/Nerd_Hour_09_13_2019/images/hrdiagram1.eps}}
    \put(30, 38){\includegraphics[height=3in]{images/Hertzsprung-Russel_StarData.eps}}
\end{picture}
\end{frame}


\begin{frame}
\frametitle{Life Cycle of a Star}
Life Cycle:
\smallskip
\begin{itemize}
    \item Massive Stars (Mass $> 8 M_{\odot}$ )
    \smallskip
        \pause
        \begin{itemize}
            \item Exist millions of years 
            \pause
            \smallskip
            \item Creates elements up to Iron (during regular burning)
            \pause
            \smallskip
            \item Core-Collapse Supernovae 
            \begin{itemize}
                \pause
                \smallskip
                \item Neutron Star
                \pause
                \smallskip
                \item Black Hole
            \end{itemize}
        \end{itemize}
    \smallskip
    \item Regular Stars (Mass $< 8 M_{\odot}$ )
        \pause
        \begin{itemize}
            \item Exist (upto) billions of years before 
            \pause
            \smallskip
            \item Creates lighter elements during regular burning (e.g. Carbon, Oxygen)
            \pause
            \smallskip
            \item Red Giant then planetary nebulae
            \pause
            \smallskip
            \item White Dwarf
            \begin{itemize}
                \pause
                \smallskip
                \item If in binary system - maybe Type Ia supernova
            \end{itemize}
        \end{itemize}
\end{itemize}
\end{frame}


\begin{frame}
\frametitle{Life Cycle of a Star}
\begin{picture}(320,250) 
    % https://www.schoolsobservatory.org/learn/astro/stars/cycle
    \put(0, 78){\includegraphics[height=2.25in]{images/starcycle.eps}}
    %\put(-10, 108){\includegraphics[height=1.75in]{/Users/asnedden/Projects/NCH/Documents/Talks/Nerd_Hour_2_11_17/figures/Tycho-supernova-xray.eps}}     % Crab Nebula = 
\end{picture}
\end{frame}


\section{Stellar Remnants}


\begin{frame}
\frametitle{White Dwarf}
\begin{picture}(320,250) 
    %https://imagine.gsfc.nasa.gov/science/objects/dwarfs1.html
    %\put(-35, 70){\includegraphics[height=2.15in]{images/sirius.eps}}
    % By NASA, ESA, H. Bond (STScI), and M. Barstow (University of Leicester) - http://www.spacetelescope.org/images/heic0516a/, CC BY 3.0, https://commons.wikimedia.org/w/index.php?curid=477445
    \put(-35, 70){\includegraphics[height=2.15in]{images/Sirius_A_and_B_Hubble_photo.eps}}
    \put(0, 50){\begin{minipage}[t]{0.7 \linewidth}
        Sirius A $\&$ B
    \end{minipage}}
       
    \put(100, 220){\begin{minipage}[t]{0.7 \linewidth}
        \begin{itemize}
            \item No nuclear fusion occuring
            \medskip
            \pause
            \item About size of the earth and mass of the sun.
            \medskip
            \pause
            \begin{itemize}
                \item Density $\sim 10^{6} g/cm^{3}$    % 5.7x10**6
            \end{itemize}
            \medskip
            \pause 
            \item Composed of light elements (e.g. Carbon, Oxygen, etc.)
            \medskip
            \pause 
            \item Held up by electron degeneracy pressure
            \begin{itemize}
                \item Recall Pauli exclusion principle?
            \end{itemize}
            \medskip
        \end{itemize}
    \end{minipage}}
\end{picture}
\end{frame}


\begin{frame}
\frametitle{White Dwarf}
\begin{picture}(320,250) 
    %https://imagine.gsfc.nasa.gov/science/objects/dwarfs1.html
    \put(10, 60){\includegraphics[height=2.4in]{images/m4wd.eps}}
    \put(90, 240){\begin{minipage}[t]{0.7 \linewidth}
       M4 - a globular cluster
    \end{minipage}}
\end{picture}
\smallskip
\end{frame}



\begin{frame}
\frametitle{Neutron Star}
\begin{picture}(320,250) 
%    % By NASA, ESA, H. Bond (STScI), and M. Barstow (University of Leicester) - http://www.spacetelescope.org/images/heic0516a/, CC BY 3.0, https://commons.wikimedia.org/w/index.php?curid=477445
%    %\put(-35, 70){\includegraphics[height=2.15in]{images/sirius.eps}}
    \put(-75, 40){\includegraphics[height=1.35in]{images/crab_nebula_xray.eps}}
    \put(-75, 150){\includegraphics[height=1.35in]{images/crab_nebula_composite.eps}}
    \put(0, 140){\begin{minipage}[t]{0.7 \linewidth}
        {\small Crab Nebula - 1054}
    \end{minipage}}
       
    \put(120, 220){\begin{minipage}[t]{0.7 \linewidth}
        \begin{itemize}
            \item No nuclear fusion occuring
            \medskip
            \pause
            \item Size of Chicago, mass of sun
            \medskip
            \pause
            \begin{itemize}
                \item Density $\sim 10^{14} g/cm^{3}$  % compute 1.5x10**15
            \end{itemize}
            \medskip
            \pause 
            \item Largest nuclei in the universe
            \medskip
            \pause 
            % Cite : http://www.if.ufrgs.br/hadrons/reisenegger1.pdf
            %      : http://adsabs.harvard.edu/full/2001ASPC..248..469R
            \item Magnetic fields of $10^{4}-10^{11} Tesla$
            \begin{itemize}
                \item Compare with MRI at 3 Tesla
            \end{itemize}
            \medskip
            \pause 
            \item Held up by neutron degeneracy pressure
            \medskip
            \begin{itemize}
                \item Recall Pauli exclusion principle?
            \end{itemize}
            \medskip
        \end{itemize}
    \end{minipage}}
\end{picture}
\end{frame}


\begin{frame}
\frametitle{Neutron Star}
Crab Nebula
\bigskip
\begin{itemize}
    \item Rotates 'slowly' at 30Hz
    \bigskip
    \pause
%https://www.mmto.org/node/675
    \item \href{https://www.dropbox.com/s/njw160luy9eb0kv/CrabPulsarResampled.mp4?dl=0}{\beamergotobutton{Crab Nebula Pulsar - MMT Observatory}}
\end{itemize}
\end{frame}


\begin{frame}
\frametitle{Neutron Star}
\begin{picture}(320,250) 

\end{picture}
\end{frame}

\begin{frame}
\frametitle{Black Hole}  
% Show video of MW central black hole. M87
stuff

\smallskip
\end{frame}


\section{Multi-Messenger Astronomy}
\begin{frame}
\frametitle{Detection}
stuff

\smallskip
\end{frame}




%\end{itemize}
%\end{frame}
\begin{frame}
\printbibliography
% cite B2FH paper
% cite neutron star merger paper and bh merger paper
\end{frame}

% Backup slides

%\begin{frame}
%\frametitle{Spare slide : Momentum Conservation}
%\end{frame}

% 1987a : https://chandra.harvard.edu/photo/2017/sn1987a/more.html
% Crab Nebula : Neutron star https://chandra.harvard.edu/photo/1999/0052/, https://svs.gsfc.nasa.gov/30944
% Vela Pulsar : https://svs.gsfc.nasa.gov/cgi-bin/details.cgi?aid=10426
% Elements https://courses.lumenlearning.com/astronomy/chapter/evolution-of-massive-stars-an-explosive-finish/


% To Do:
% 1. Read astronut about stars to enusre I got the physics right.
% 2. Read about nuclear fusion
% 3. Jargon alert? 
% 4. First slide should motivate talk. Why do we care about stars?
% 5. Plot of lifetime vs. mass
% 6. Look up why planetary nebula forms
% 7. Periodic Table
% 8. Bring the Bible and Astronut
% 9. Explain why neutron stars spin




\end{document}
